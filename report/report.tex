\documentclass[11pt]{article}

\usepackage[a4paper, total={16cm, 24cm}]{geometry}
\usepackage[portuguese]{babel}
\usepackage[utf8]{inputenc}
\usepackage{graphicx}
\usepackage{amsmath}
\usepackage{tikz}
    \usetikzlibrary{shadows}
\usepackage{booktabs}
\usepackage[colorlinks=true]{hyperref}
\usepackage{listings}
    \renewcommand\lstlistingname{Listagem}
    \lstset{numbers=left, numberstyle=\tiny, numbersep=5pt, basicstyle=\footnotesize\ttfamily, frame=tb,rulesepcolor=\color{gray}, breaklines=true}
\usepackage{blindtext}

% -------------------------------------------------------------------------------------------
\title
{
    \includegraphics[width=0.4\textwidth]{imgs/university.png}
    \\[0.1cm]
    \textbf{1ª Versão} \\
    Sistemas Móveis e Aplicações
}

\author
{
    \textbf{Professor:} Vítor Nogueira \\
    \textbf{Realizado por:} Miguel de Carvalho (43108), Filipe Alfaiate (43315)
}
\date{\today}

% -------------------------------------------------------------------------------------------
%                                Body                                                       %
% -------------------------------------------------------------------------------------------

\begin{document}
\maketitle

% -------------------------------------------------------------------------------------------
\begin{itemize}
    \item Nome da aplicação: \textbf{Virtual Tripmeter}
    \item Objetivos: 
    \begin{itemize}
        \item Ajudar os Co-Pilotos de \textbf{Rally} a conseguirem treinar a leitura
        dos \textbf{RoadBooks} (indicações do caminho), simulando o funcionamento do
        \textbf{TerraTrip}. É utilizado para registar as distâncias percorridas (Parciais e
        Totais), que auxilia na leitura do \textbf{RoadBook}, pois as \textbf{notas} apresentam
        sempre uma \textbf{distância parcial} entre elas e apresenta também a \textbf{distância 
        total} desde o início até a posição atual do carro.
    \end{itemize}
    \item Características:
    \begin{itemize}
        \item Simulação da estrutura de um \textbf{TerraTrip}, ou seja, 2 contadores de
        de unidade de distância calculados pela "velocidade" a que o veículo se desloca.
        A caixas podem ser \textit{resetadas} com um \textit{long press} ou com um \textit{tap};
        \item Possibilidade de aumentar ou diminuir a velocidade atual de forma rápida ou lenta,
        consoante o toque nos \textbf{botões};
        \item Possibilidade de \textbf{começar} (play), \textbf{parar} (pause) e \textbf{terminar} (stop) a simulação;
        \item Possibilidade de usar \textbf{unidades imperiais} (Milhas) em vez de 
        \textbf{unidades métricas} (Quilómetros);
        \item Escolher uma velocidade base;
        \item Possibilidade de emitir um alerta quando é realizado um \textit{reset} num
        dos contadores.
    \end{itemize}
\end{itemize}

% -------------------------------------------------------------------------------------------
\end{document}